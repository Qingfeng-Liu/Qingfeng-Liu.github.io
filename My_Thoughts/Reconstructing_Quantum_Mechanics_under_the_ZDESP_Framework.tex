\documentclass[12pt,a4paper]{article}
\usepackage{amsmath,amssymb,geometry}
\geometry{margin=1in}
\usepackage{hyperref}

\begin{document}
	
	\title{Reconstructing Quantum Mechanics under the ZDESP Framework: From Unfalsifiable Claims to Experimentally Testable Models}
	\author{}
	\date{}
	\maketitle
	
	\section{Quantum Measurement and the Unfalsifiability of the ``State Change'' Proposition}
	
	Consider a quantum experiment characterized by three components: the initial state $\rho$; the unitary time evolution $U_t$ in the absence of observation (generated by the Hamiltonian); and the \emph{quantum instrument} $\{\mathcal M_k\}_k$ corresponding to the measurement device. Each branch $\mathcal M_k$ of the instrument is a completely positive map that sends a density matrix to a subnormalized state, inducing the outcome probability and the post-measurement state:
	\[
	p_k = \operatorname{Tr}[\mathcal M_k(\rho)], \qquad
	\sigma_k = \frac{\mathcal M_k(\rho)}{p_k}.
	\]
	The POVM element associated with $\mathcal M_k$ is $E_k = \mathcal M_k^{\ast}(\mathbf 1)$, so that
	\[
	p_k = \operatorname{Tr}(E_k \rho).
	\]
	The only experimentally accessible quantities are the joint distributions of outcomes under various measurements (single or sequential). For example, for a two-step measurement:
	\[
	p_{k,\ell} = \operatorname{Tr}\!\big[ \mathcal M_\ell^{(2)} \circ \mathcal M_k^{(1)} (\rho) \big].
	\]
	
	Intuitively, the ``joint distribution'' can be thought of as follows: one performs the same experiment many times, records the sequence of outcomes for each run, and then compiles the frequencies of these sequences. All observable quantities depend only on the \emph{external action} of the instrument (the $\mathcal M_k$ as a family of maps), i.e., the outcome distribution it generates and the way these distributions behave under sequential composition.
	
	The claim that ``measurement changes the state'' is usually formulated as ``collapse of the wavefunction'': a jump from the unmeasured evolution $U_t \rho U_t^\dagger$ to some $\sigma_k$. To make this a falsifiable proposition, one must compare the \emph{same physical system}'s counterfactual state without measurement,
	\[
	\rho_{\mathrm{no\,meas}}(t) = U_t \rho U_t^\dagger,
	\]
	with the conditional state $\sigma_k(t)$ given that a measurement at that moment yielded outcome $k$.
	
	The problem is that these are \emph{mutually exclusive counterfactuals}. In any given run, you either perform the measurement (thus obtaining $\sigma_k$ and its subsequent statistics) or you do not (thus obtaining $\rho_{\mathrm{no\,meas}}$ and its subsequent statistics). You cannot have both in the same world-line. Therefore, ``whether the measurement changes the state''—for the same system—is an \textbf{unidentifiable counterfactual} in the causal–statistical sense.
	
	From the viewpoint of \emph{representation invariance}, the notion of ``change'' is not even a representation-independent statement. By Stinespring/Naimark dilation, one can regard the system $S$ and the measurement pointer $A$ as a combined system:
	\[
	\rho_S \ \longmapsto \ U_{SA} \big( \rho_S \otimes |0\rangle\!\langle 0|_A \big) U_{SA}^\dagger,
	\]
	followed by a projection $\{\Pi_k\}_A$ on the pointer and conditionalization. In this picture, the total system $S+A$ evolves unitarily at all times, without any ``collapse''; the ``collapse'' is merely the conditioning of the subsystem $S$:
	\[
	\sigma_k = \frac{\operatorname{Tr}_A\!\Big[ (\mathbf 1 \otimes \Pi_k) \, U_{SA} (\rho_S \otimes |0\rangle\!\langle 0|_A) U_{SA}^\dagger \Big]}{p_k}.
	\]
	Whether we say the state ``changes'' depends entirely on where we place the Heisenberg cut. In other words, ``measurement changes the state'' is not a cut-invariant statement about the world; it is a matter of how we choose to partition the system and the measurement apparatus. Changing the partition changes the description of ``change'', but leaves all observable joint probabilities $p_{k_1,\ldots,k_n}$ unchanged.
	
	Intuitively, this is like describing the same play in two languages: in one, the actor ``changes costume'' (collapse); in the other, the actor has been wearing the full costume all along, but we only see part of the stage (conditionalization). The audience hears the same lines (observable probabilities) in either description.
	
	Some might argue: we can insert an intermediate measurement in a sequence, and if the later statistics change, that proves the measurement disturbed the state. Strictly speaking, this only characterizes the disturbance properties of the instrument: different $\mathcal M_k$ compositions yield different joint distributions $p_{k,\ell}, p_{k,\ell,m}, \dots$. You \emph{can} falsify a particular \emph{no-disturbance model} for a given instrument, but you cannot falsify the general claim that ``measurement changes the state'', because the same joint distributions can be generated by a ``larger-system unitary + conditioning'' description, in which nothing in the global state changes—only the conditioned subsystem changes. The two narratives make exactly the same predictions for all observable statistics.
	
	In conclusion, within the standard formalism of quantum theory, the accessible experimental data are the outcome distributions and their sequential compositions, determined by the instrument. The ``change'' in ``measurement changes the state'' cannot be identified via single-world-line counterfactuals, nor is it a representation-invariant fact about the total system. It can always be rephrased as ``unitary evolution + conditioning'', with both narratives yielding identical predictions for all observable statistics. Therefore, treating ``measurement changes the state'' as an independent, falsifiable physical statement is impossible: it is a choice of narrative for the same set of observable distributions, not an additional empirical content.
	
	\section{Observation and State: From Unfalsifiable to ZDESP}
	
	\subsection{The Problem of Traditional Quantum Mechanics Statements}
	In the Copenhagen interpretation of traditional Quantum Mechanics (QM), observation is believed to cause a change in the state of a quantum system, i.e., the collapse of the wave function. Mathematically, this is formulated as a measurement operation projecting the system into one of the eigenstates of the measurement operator. However, from the perspective of philosophy of science, this statement has a fundamental problem—it is unfalsifiable.
	
	The reason is that it is impossible to know the state before measurement without performing a measurement. To compare the difference between “before observation” and “at the moment of observation,” one must acquire information at both moments, but acquiring information is itself a measurement, which introduces the observation effect. Thus, this claim becomes one that can never be experimentally refuted.
	
	Under such circumstances, the so-called “observation changes the state” is not a physical law with testable causality, but rather an interpretative hypothesis of observed data.
	
	\subsection{Restatement from a Statistical Perspective}
	From the viewpoint of statistics, this problem is equivalent to:
	\begin{itemize}
		\item We are dealing with a latent variable model, where the “true state” is an unobservable latent variable;
		\item We can only obtain realizations of the measurement random variable;
		\item We cannot obtain the true distribution before measurement to verify whether measurement changes it.
	\end{itemize}
	
	Therefore, “measurement changes the state” and “the state is inherently random” may be statistically indistinguishable. Their difference exists only at the level of model assumptions, not within the realm of verifiable observation data.
	
	\subsection{Introduction of ZDESP}
	To break this unfalsifiable deadlock, we propose the \textbf{ZDESP (Zero Distance in Extended Space Paradigm)}. The core ideas are:
	\begin{itemize}
		\item Quantum objects do not exist solely in three-dimensional space plus time (spacetime), but simultaneously reside in a higher-dimensional extended space;
		\item In the extended space, certain quantum states have ``zero distance'' between them, even if they are separated by billions of light-years in spacetime;
		\item Observation is merely a local projection operation in the extended space, rather than a physical transmission process, and thus is not constrained by the speed-of-light limit nor requires the assumption of instantaneous action or nonlocal entanglement.
	\end{itemize}
	
	In this framework, observation does not “change” the state, but \textbf{selects} a projectable state in the extended space to map into our spacetime observational coordinates. State evolution in the extended space is continuous and independent, not dependent on the act of observation itself.
	
	\subsection{Theoretical Significance and Experimental Direction}
	The advantages of ZDESP are:
	\begin{enumerate}
		\item It eliminates the unfalsifiable claim that “measurement changes the state”;
		\item It provides a physical-mathematical model for defining “zero distance” in higher-dimensional space, making nonlocal phenomena interpretable without invoking instantaneous action;
		\item It offers a new geometric perspective for phenomena such as quantum information transfer and entanglement preservation.
	\end{enumerate}
	
	The subsequent chapter will propose experimental schemes under the ZDESP framework to test its statistical distinguishability from traditional QM.
	
	\section{Experiment Design for Falsifiability between ZDESP and Traditional QM}
	
	\subsection{Objective}
	The goal is to identify, under the same physical conditions, a statistical feature of a measurable quantity such that:
	\begin{itemize}
		\item If the experimental result satisfies condition $R_1$, ZDESP is supported;
		\item If the experimental result satisfies condition $R_2$, traditional QM is supported;
		\item The predicted differences between the two exceed the statistical significance threshold over experimental noise and systematic error.
	\end{itemize}
	
	\subsection{Candidate Experiment Type}
	A candidate scheme is based on delayed choice entanglement swapping (DCES):
	\begin{enumerate}
		\item Generate two entangled pairs $(A,B)$ and $(C,D)$;
		\item Perform a delayed Bell-state measurement (BSM) between $B$ and $C$;
		\item Measure the correlation between $A$ and $D$ as the timing of BSM changes.
	\end{enumerate}
	
	\subsection{Predicted Differences}
	\begin{itemize}
		\item \textbf{Traditional QM}: Regardless of whether BSM occurs before or after the measurement of $A$ and $D$, the correlation statistics (such as the degree of CHSH inequality violation) should remain identical.
		\item \textbf{ZDESP}: Since $A$ and $D$ have zero-distance connection in the extended space independent of timing, changes in BSM timing may cause slight systematic variations in certain statistical patterns, specifically manifesting as correlation coefficient shifts within certain delay ranges.
	\end{itemize}
	
	\subsection{Decision Criteria}
	\begin{itemize}
		\item If the correlation curves coincide under different delay conditions (within statistical error), traditional QM is supported;
		\item If systematic deviations beyond statistical error appear, matching the pattern predicted by ZDESP, then ZDESP is supported.
	\end{itemize}
	
	\subsection{Feasibility and Challenges}
	The experiment requires extremely high time resolution, long entanglement coherence time, and ultra-low-noise detection systems. Its feasibility depends on upgrades to existing quantum optics experimental platforms.
	
\end{document}